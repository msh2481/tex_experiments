\documentclass[12pt]{article}

% Cyrillic support
\usepackage{polyglossia}
\setmainfont{XCharter}
\usepackage{xcharter-otf}
\usepackage{parskip}

% Images
\usepackage[dvips,xetex]{graphicx}
\graphicspath{ {./data/} }
% Spaces after sentence ends
\frenchspacing
% Advanced coloring
\usepackage{xcolor}
% Math environments
\usepackage{amsthm}
% \usepackage{amssymb}
\newtheorem{thr}{Th.}
\newtheorem{alg}{Alg.}
\theoremstyle{definition}
\newtheorem*{df}{Def}
% One-symbol defines with mathbb
\usepackage{amsfonts}
\providecommand{\N}{\mathbb N}
\providecommand{\Z}{\mathbb Z}
\providecommand{\P}{\mathbb P}
\providecommand{\R}{\mathbb R}
\providecommand{\C}{\mathbb C}
\providecommand{\F}{\mathbb F}
\providecommand{\Q}{\mathbb Q}
\providecommand{\bigO}{\mathcal O}
\providecommand{\go}[1]{\url{#1.pdf}}
\providecommand{\then}{\Rightarrow}
\providecommand{\from}{\Leftarrow}
\providecommand{\iff}{\Leftrightarrow}

\renewcommand{\N}{\mathbb N}
\renewcommand{\Z}{\mathbb Z}
\renewcommand{\P}{\mathbb P}
\renewcommand{\R}{\mathbb R}
\renewcommand{\C}{\mathbb C}
\renewcommand{\F}{\mathbb F}
\renewcommand{\Q}{\mathbb Q}
\renewcommand{\bigO}{\mathcal O}
\renewcommand{\go}[1]{\url{#1.pdf}}
\renewcommand{\then}{\Rightarrow}
\renewcommand{\from}{\Leftarrow}
\renewcommand{\iff}{\Leftrightarrow}

\renewcommand{\le}{\leqslant}
\renewcommand{\ge}{\geqslant}
\usepackage{physics}
% Code listings
\usepackage{minted}
\usepackage{hyperref}
\hypersetup{
	linkcolor=.,
	colorlinks=true,
	urlcolor=cyan
}
\renewcommand{\go}[1]{\href{./#1.pdf}{#1}}
\providecommand{\mnt}[1]{\mintinline{python}{#1}}
\title{Higher derivatives}
\author{}
\date{01 May 2022}
\begin{document}
\maketitle
\section*{Simplest case: $\R^m \to \R$}
Consider $f: \R^m \to \R$, differentiable on $\R^m$. Then $\exists \pdv{x_i}: \R^m \to \R$.
If it's again differentiable, $\exists \pdv{x_j} \pdv {x_i}: \R^m \to \R$. And so on.

$ \frac{\partial^p f}{\partial x_{i_1} \dots \partial x_{i_p}} $ denotes the derivative of $p$-th order.

\begin{thr}[Order of differentiation]
     If $\pdv{f}{x_i}{x_j}, \pdv{f}{x_i}{x_j}$ 
    exist in a neighborhood and are continuous, they are equal.
\end{thr}
\begin{proof}
    For simplicity, $i = 1, j = 2, m = 2$. Also consider the discrete derivative:
    \[F(h_1, h_2) = (f(x_1 + h_1, x_2 + h_2) - f(x_1 + h_1, x_2)) - (f(x_1, x_2 + h_2) - f(x_1, x_2))\]
    And another helper function:
    \[ \varphi(t) = f(x_1 + h_1, x_2 + th_2) - f(x_1, x_2 + th_2) \]
    Let's use mean value theorem along $x_2$:
    \[ F(h_1, h_2) = \varphi(1) - \varphi(0) = \varphi'(\theta_2) \]
    \[ \varphi'(t) = h_2(\pdv{f}{x_2}\/(x_1 + h_1, x_2 + th2) - \pdv{f}{x_2}\/(x_1, x_2 th2)) \]
    Now use it again along $x_1$:
    \[\varphi'(t) = h_1 h_2 \pdv{f}{x_1}{x_2}(x_1 + \theta_1 h_1, x_2 + \theta_2 h_2) \]
    \[ F(h_1, h_2) = h_1 h_2 \pdv{f}{x_1}{x_2}(x_1 + \theta_1 h_1, x_2 + \theta_2 h_2), \exists \theta_1, \theta_2 \in (0, 1) \]
    \[ \lim\limits_{h\to 0} \frac{F(h_1, h_2)}{h_1 h_2} = \pdv{f}{x_1}{x_2}\/(x) \text{ (by continuity of } \pdv{f}{x_1}{x_2} \text{)}\]
    Similarly, we can do it in other order and get:
    \[ \lim\limits_{h\to 0} \frac{F(h_1, h_2)}{h_1 h_2} = \pdv{f}{x_2}{x_1}\/(x) \]
\end{proof}

\begin{thr}[Order of differentiation, corollary]
    If a function has continuous derivatives of $n$-th order, they don't depend on the order of differentiation.
\end{thr}
\begin{proof}
    Suppose there are two orders for which the results differ, and the differnce consists of swapping two neighbors. 
    All other permutations can be derived from it.
    
    We can remove their common prefix, because after differentiating it we got the same functions. Here we apply the previous theorem, so 
    we get the same functions again. And that remains is their suffixes, which are equal too.
\end{proof}

\section*{More complicated case: $X \to Y$}
Consider $f: X \to Y$, differentiable on $X$. Then $\exists \dd{f}: X \to L(X, Y)$.
If it's again differentiable, $\exists \dd(\dd{f}): X \to L(X, X, Y)$ (or $L(X, L(X, Y))$). 
It is the second differential and can be denoted $\dd[2]{f}$. If it's differentiable too we can go on further.

\begin{thr}[Relation of the first differential to directed derivatives]
    \[\dd{f}(x)h = \pdv{f}{h}\/(x) = \lim\limits_{t \to 0} \frac{f(x + th) - f(x)}{t}\]
\end{thr}

\begin{thr}[Relation of the second differential to directed derivatives]
    \[\dd[2]{f}(x)(h_1, h_2) = \dd(\dd{f})(x)(h_1, h_2) = (\pdv{h_1}(\dd{f}))(x)(h_2) \]
    \[ = \qty(\lim\limits_{t \to 0} \frac{\dd{f}(x + th_1) - \dd{f}(x)}{t})(h_2) = \lim\limits_{t \to 0} \frac{\dd{f}(x + th_1)h_2 - \dd{f}(x)h_2}{t} \]
    \[ = \lim\limits_{t \to 0} \frac{\pdv{f}{h_2}(x+th_1) - \pdv{f}{h_2}(x)}{t} = \pdv{h_1}\qty(\pdv{f}{h_2})(x) \]
\end{thr}
And we can continue applying it by induction.

\begin{thr}[Symmetricity of differential form]
    If $\exists \dd[n]{f}(x)$ and is continuous, then the order of its arguments doesn't matter.
\end{thr}
\begin{proof}
    Let's use the form of directed derivative and consider $n = 2$. So we want to prove that:
    \[ \pdv{h_1} \pdv{h_2} f(x) = \pdv{h_2} \pdv{h_1} f(x) \]
    Again let's introduce some helper functions:
    \[ F(h_1, h_2, t) = f(x + th_1 + th_2) - f(x + th_1) - f(x + th_2) + f(x) \] 
    \[ \varphi(v) = f(x + th_1 + tv) - f(x + tv), v \in [0, h_2] \] 
    \[ F(h_1, h_2, t) = \varphi(h_2) - \varphi(0) \]
    Now we'll show $\norm{F(h_1, h_2, t) - t^2 \dd[2]{f}(x)(h_1, h_2)} = o(1)$:
    \[ \norm{\varphi(h_2) - \varphi(0) - t^2 \dd[2]{f}(x)(h_1)(h_2)} \leqslant
      \sup\limits_{\theta_2 \in (0, 1)} \norm{\dd{\varphi(\theta_2 h_2) - t^2\dd[2]{f}(x)(h_1)} } \norm{h_2} = \dots \]
    \[ \qty( \dd{\varphi}(\theta_2 h_2) = t(\dd{f}(x + th_1 + t \theta_2 h_2) - \dd{f}(x + t \theta_2 h_2)) ) \] 
    \[ \dots = \abs{t}\norm{h_2} \sup \norm{\dd{f}(x + \theta_2 h_2 t + th_1) - \dd{f}(x + \theta_2 h_2 t) - t \dd[2]{f}(x)(h_1)} \leqslant \]
    \[ \abs{t}\norm{h_2} \sup \sup \norm{ \dd[2]{f}(x + \theta_2 h_2 + \theta_1 h_1) - \dd[2]{f}(x)}\norm{th_1} = o(t^2) \]

\end{proof}

\section*{Relation of differential to the partial derivatives}
Consider $f: \R^n \to \R$, its $\dd[n]{f}$ is a polylinear map, therefore can be expressed as:
\[ \dd[n]{f}(x)(h_1 \dots h_n) = \sum\limits_{1 \leqslant j_1 \dots j_n \leqslant n} \dd[n]{f}(x)(e_{j_1} \dots e_{j_n}) h_1^{(j_1)} \dots h_n^{(j_n)} \]
From that we get that $n$-th differential is a polylinear map and its coefficients are the partial derivatives.

\section*{Taylor formula}
Again, $f: \R^m \to \R, f\in C^{n+1}[x, x + h]$, then (here $\dd[k]{f}(x) h^k$ means substitution of $k$ entries of $h$):
\[ f(x + h) = f(x) + \sum\limits_{k=1}^n \frac{\dd[k]{f}(x) h^k}{k!} + \frac{\dd[n + 1]{f}(x + \theta h)h^{k + 1}}{(k + 1)!} \]
\begin{proof}
\[\varphi(t) = f(x + th), \varphi \in C^{n+1}[0, 1]\]
\[\varphi(1) = \varphi(0) + \sum\limits_{k=1}^n \frac{\varphi^{(k)}(0)}{k!} + \frac{\varphi{(k+1)}(\theta)}{(k + 1)!} \]
\end{proof}
\end{document}