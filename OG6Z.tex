\documentclass[12pt]{article}

% Cyrillic support
\usepackage{polyglossia}
\setmainfont{XCharter}
\usepackage{xcharter-otf}
\usepackage{parskip}

% Images
\usepackage[dvips,xetex]{graphicx}
\graphicspath{ {./data/} }
% Spaces after sentence ends
\frenchspacing
% Advanced coloring
\usepackage{xcolor}
% Math environments
\usepackage{amsthm}
% \usepackage{amssymb}
\newtheorem{thr}{Th.}
\newtheorem{alg}{Alg.}
\theoremstyle{definition}
\newtheorem*{df}{Def}
% One-symbol defines with mathbb
\usepackage{amsfonts}
\providecommand{\N}{\mathbb N}
\providecommand{\Z}{\mathbb Z}
\providecommand{\P}{\mathbb P}
\providecommand{\R}{\mathbb R}
\providecommand{\C}{\mathbb C}
\providecommand{\F}{\mathbb F}
\providecommand{\Q}{\mathbb Q}
\providecommand{\bigO}{\mathcal O}
\providecommand{\go}[1]{\url{#1.pdf}}
\providecommand{\then}{\Rightarrow}
\providecommand{\from}{\Leftarrow}
\providecommand{\iff}{\Leftrightarrow}

\renewcommand{\N}{\mathbb N}
\renewcommand{\Z}{\mathbb Z}
\renewcommand{\P}{\mathbb P}
\renewcommand{\R}{\mathbb R}
\renewcommand{\C}{\mathbb C}
\renewcommand{\F}{\mathbb F}
\renewcommand{\Q}{\mathbb Q}
\renewcommand{\bigO}{\mathcal O}
\renewcommand{\go}[1]{\url{#1.pdf}}
\renewcommand{\then}{\Rightarrow}
\renewcommand{\from}{\Leftarrow}
\renewcommand{\iff}{\Leftrightarrow}

\renewcommand{\le}{\leqslant}
\renewcommand{\ge}{\geqslant}
\usepackage{physics}
% Code listings
\usepackage{minted}
\usepackage{hyperref}
\hypersetup{
	linkcolor=.,
	colorlinks=true,
	urlcolor=cyan
}
\renewcommand{\go}[1]{\href{./#1.pdf}{#1}}
\providecommand{\mnt}[1]{\mintinline{python}{#1}}
\title{Implicit and inverse function theorem}
\author{}
\date{05 May 2022}
\begin{document}
\maketitle

\begin{thr}[Implicit function theorem]
$X, Y, Z$ --- normed, $Y$ --- full,\\
$W \subset X \times Y$ --- open, $(x_0, y_0) \in W$,\\
$G: W \to Z$, $G$ continuous at $(x_0, y_0)$, $G(x_0, y_0) = 0$,\\
$\exists \partial_yG$ in $W$ and it's continuous at $(x_0, y_0)$,\\
$\exists (\partial_y G\/(x_0, y_0))^{-1} \in L(Z, Y).$\\
Then $\exists U, V$: neighborhoods of $x_0 y_0$ and 
$f: U \to V$ continuous at $x_0$ such that $G(x, f(x)) = 0$.
\end{thr}
\begin{proof}
Let's use Newton's method to find $y$:
\[ g_x: y \mapsto y - (\partial_yG(x_0, y_0))^{-1}G(x, y) \]
\[ g_x: Y \to Y \]
Consider it's differential:
\[ \dd{g_x}\/(y) = I_y -  (\partial_yG(x_0, y_0))^{-1}\dd{G}\/(x, y) \]
\[ (G(x, y) \to G(x_0, y_0)) \to (\dd{g_x} \to 0) \]
Then in some neighborhood of $(x_0, y_0)$ $\norm{\dd{g_x}\/(y)} < \frac12$.
Also in some neighborhood of $x_0$: $\norm{g_x(y_0) - g_{x_0}(y_0)} < \varepsilon$ by continuity of $G$.
For the chosen $\varepsilon < \frac{\Delta}{2}$ and $x$ in the chosen $\delta$-neighborhood, with $L_{+\infty}$ norm:
\[ g_x(B(y_0, 2\varepsilon)) \subset B(y_0, 2\varepsilon) \]
Let's prove it:
\[\norm{g_x(y) - y_0} = \norm{g_x(y)  g_{x_0}(y_0)} \le\]
\[\norm{g_x(y) - g_x(y_0)} + \norm{g_x(y_0) - g_{x_0}(y_0)} \le\]
\[\sup\norm{\dd{g_x}} \norm{y - y_0} + \varepsilon \le \frac12 \varepsilon + \varepsilon\]
So now we take $x$ in $B(x_0, \min(\delta, \Delta))$, $y$ start from$y_0$ and stay in $B(y_0, \Delta)$,
therefore $g_x$ is squeezing and we finally arrive at unique $y$ where $G(x, y) = 0$.

And $f$ is continuous at $x_0$ because we can make $\varepsilon$ smaller by taking $\delta$ smaller.
\end{proof}

\subsection*{Basic example}
\[ \begin{cases}
    g_1(x_1 \dots x_m, y_1 \dots y_n) = 0\\
    \dots \\
    g_n(x_1 \dots x_m, y_1 \dots y_n) = 0\\
\end{cases} \]
\[ \leftrightarrow \] 
\[ \begin{cases}
    y_1 = f_1(x_1 \dots x_m)\\
    \dots \\
    y_n = f_1(x_1 \dots x_m)\\
\end{cases} \]
It's a surface with dimension $m$ (because every point is determined by it's $x$ coordinates) in $\R^{n+m}$.

\begin{thr}
$Y$ --- complete normed space, $U \in L(Y, Y)$, $\norm{U} < 1$, I --- identity\\
$ \to \exists (I - U)^{-1} \in L(Y, Y) $
\end{thr}
\begin{proof} First proof.

\par Existence.

Want to prove that $\forall u \in Y \exists y \in Y: (I - U)y =  u$.
\[ y_{n+1} = u + Uy_n \] 
\[ y_{n+1} - y_n = Uy_n - Uy_{n-1} \] 
\[ \norm{ y_{n+1} - y_n } = \norm{U(y_n - y_{n-1})} \le \norm{U}\norm{y_n - y_{n-1}} \] 
If we iterate this squeeze mapping, we will get the unique solution $y_0$ (here we used completeness of $Y$).

\par Continuity.

Now we consider $u_n \to u_0$, for each we find $y_n$ and want to show $y_n \to y_0$.
\[ (I - U)y_n = u_n, (I - U)y_0 = u_0 \then (y_n - y_0) = U(y_n - y_0) + (u_n - u_0)\]
\[ \norm{y_n - y_0} \le \norm{u_n - u_0} + \norm{U} \norm{y_n - y_0} \]
\[ 0 \le (1 - \norm{U})\norm{y_n - y_0} \le \norm{u_n - u_0} \to 0 \]
\[ 1 - \norm{U} > 0 \then \norm{y_n - y_0} \to 0 \]

\end{proof}

\begin{proof} Second proof. 

\[ (I - U)^{-1} = I + U + U^2 + \dots \] 

This series converges absolutely, i.e. $\sum \norm{U^k} \le \frac{1}{1 - \norm{U}}$ converges.

$L(Y, Y)$ is complete (link?), so every Cauchy sequence converges. And absolutely converging series are Cauchy sequences.

Now, consider $S$:
\[ S_n = I + U + \dots + U^n \to S \in L(Y, Y) \] 
\[ S_n(I - U) = (I - U)S_n = I - U^{n+1} \to I \] 
\[ S(I - U)  = (I - U)S = I \to S = (I - U)^{-1} \]

\end{proof}

\textit{Interesting fact} $A \in L(X, Y), B \in L(Y, X), AB = I_x$, then:
\[ \frac{1}{\norm{B}} = \inf\limits_{\norm{x}=1} \frac{1}{\norm{Bx}} = 
 \inf\limits_{\norm{Bx} = 1} \frac{\norm{x}}{\norm{Bx}} = 
\inf\limits_{\norm{y}=1} \frac{\norm{Ay}}{\norm{y}} = \inf\limits_{\norm{y}=1} \norm{Ay} \]

\begin{thr} 
    \[ Y\text{ --- complete }, U \in L(Y, Z), \exists U^{-1} \in L(Z, Y) \forall V \in L(Y, Z) \] 
    \[ \norm{V} < \dfrac{1}{\norm{U^{-1}}} \to \exists (U \pm V)^{-1} \in L(Z, Y) \]
\end{thr}
\begin{proof}
    \[U + V = U(I + U^{-1}V)\]
    \[(U+V)^{-1} = (I +  U^{-1}V)^{-1} U^{-1}\]
    Now use completeness of $Y$ and the previous theorem:
    \[\norm{U^{-1}V} < 1 \then \exists (I + UV^{-1})^{-1}\]
\end{proof}

\begin{thr}
    As Th. 1, but require continuity of $G$ and $\dd{G}$ not only in the point,
    but in a neighborhood. Then $f$ will be continuous in a neighborhood of $x_0$.
\end{thr}

\begin{proof}
    Consider $\Delta$-neighborhood from Th. 1 where $f$ exists. 
    If we take $(x_1, y_1)$  from there then
     $\exists (\partial_y G(x_1, y_1))^{-1} \in L(Z, Y)$ by previous lemma, so 
     we can apply Th. 1 to that point and $f$ will be the same, but now with 
     continuity at that point too.

     Note that neighborhood where it holds might be smaller than $\Delta$ because 
     we didn't have any requirements on where $\exists (\partial_y G(x_1, y_1))^{-1} \in L(Z, Y)$.
\end{proof}

\begin{thr}
    As Th. 1, but $\exists \dd{G}$ then $\exists \dd{f}$ and:
    \[ \dd{f}\/(x_0) = -(\pdv{y} G(x_0, y_0))^{-1}\pdv{x}G(x_0, y_0) \] 
    and all three parts are in $L(X, Y), L(Z, Y), L(X, Z)$ correspondingly.
\end{thr}

\begin{proof}
    \[ G(x, y) = G(x_0, y_0) + \pdv{x}G(x_0, y_0)(x-x_0) + \pdv{y}G(x_0, y_0)(y - y_0) + o\qty(\norm{x - x_0} + \norm{y - y_0}) \]
    Consider $y = f(x)$ (and $G(x, y) = 0$ now):
    \[ 0 = \pdv{x}G(x_0, y_0)(x-x_0) + \pdv{y}G(x_0, y_0)\qty(f(x) - f(x_0)) + o\qty(\norm{x - x_0} + \norm{f(x) - f(x_0)}) \] 
    Divide by $\pdv{y}G(x_0, y_0)$ and find $f(x) - f(x_0)$:
    \[ f(x) - f(x_0) = -\qty(\pdv{y}G(x_0, y_0))^{-1} \qty(\pdv{x}G(x_0, y_0) (x - x_0) + o\qty(\norm{x - x_0} + \norm{f(x) - f(x_0)})) \] 

    Now we need to show that $o(\dots)$ is small enough and $f$ is continuous:

    Let $C_1 = \norm{\qty(\pdv{y}G(x_0, y_0))^{-1}} \norm{\pdv{x}G(x_0, y_0)}, C_2 = \norm{\qty(\pdv{y}G(x_0, y_0))^{-1}}, \eps \to 0 \text{ from } o(\dots)$:
    \[ \norm{y - y_0} \le C_1 \norm{x - x_0} + C_2 \eps \qty(\norm{x - x_0} + \norm{y - y_0}) \] 
    Then for $\eps < C_2^{-1}$:
    \[ \norm{y - y_0} \le \frac{C_1 + C_2 \eps}{1 - C_2 \eps} \norm{x - x_0} \]
\end{proof}

\begin{thr}[Corollary] 
    If $G$ is $k$ times differentiable in a neighborhood, then $f$ is too.
\end{thr}

\begin{thr}[Inverse function theorem]
    $Y$ --- complete, $F: Y \to X, F(y_0) = x_0, \exists \dd{F}$ in a neighborhood, $\exists (\dd{F}\/(y_0))^{-1} \in L(X, Y)$ \\ 
    then exist neighborhoods $U, V: x_0 \in U, y_0 \in V$ such that $F: V \to U$ --- bijection and
    \[ (\dd{F^{-1}})(x_0) = (\dd{F}\/(y_0))^{-1} \]  
\end{thr}
\begin{proof}
    \[ G: X \times Y \to X: G(x, y) = x - F(y) \]
    It's suitable for Th. 1:
    \[ \pdv{y} G\/(x_0, y_0) = -\dd{F}, G(x_0, y_0) = 0 \]
    Then by using it we get $f(x)$:
    \[ \exists U, V \forall x \in U, y \in V : x = F(y) \leftrightarrow G(x, y) = 0 \leftrightarrow y = f(x) \]
    Now it's already a map and even an injection, but we want to make $V = F(U)$ to make it a bijection. And we need $y_0$ to be internal in it.
    \[ \dd{f}\/(x_0) = -(\pdv{y} G(x_0, y_0))^{-1} \pdv{x}\/G(x_0, y_0) = (\dd{F}\/(y_0))^{-1} \]
    \[ \exists (\dd{f}\/(x_0))^{-1} = \dd{F}\/(y_0) \in L(Y, X) \] 
    So for any point close to $y_0$ its preimage is close to $x_0$, and $y_0$ is internal in $F(U)$.
    Now $f$ and $F$ are mutually inverse and $V = F(U), U = f(V)$, so they are bijections.
\end{proof}

\end{document}