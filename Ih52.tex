\documentclass[12pt]{article}

% Cyrillic support
\usepackage{polyglossia}
\setmainfont{XCharter}
\usepackage{xcharter-otf}
\usepackage{parskip}

% Images
\usepackage[dvips,xetex]{graphicx}
\graphicspath{ {./data/} }
% Spaces after sentence ends
\frenchspacing
% Advanced coloring
\usepackage{xcolor}
\usepackage{hyperref}
\hypersetup{
	linkcolor=.,
	colorlinks=true,
	urlcolor=cyan
}
% Math environments
\usepackage{amsthm}
% \usepackage{amssymb}
\newtheorem{thr}{Th.}
\newtheorem{alg}{Alg.}
% One-symbol defines with mathbb
\usepackage{amsfonts}
\providecommand{\N}{\mathbb N}
\providecommand{\Z}{\mathbb Z}
\providecommand{\P}{\mathbb P}
\providecommand{\R}{\mathbb R}
\providecommand{\C}{\mathbb C}
\providecommand{\F}{\mathbb F}
\providecommand{\Q}{\mathbb Q}
\providecommand{\bigO}{\mathcal O}
\providecommand{\go}[1]{\url{#1.pdf}}
\renewcommand{\N}{\mathbb N}
\renewcommand{\Z}{\mathbb Z}
\renewcommand{\P}{\mathbb P}
\renewcommand{\R}{\mathbb R}
\renewcommand{\C}{\mathbb C}
\renewcommand{\F}{\mathbb F}
\renewcommand{\Q}{\mathbb Q}
\renewcommand{\bigO}{\mathcal O}
\renewcommand{\go}[1]{\url{#1.pdf}}
\usepackage{physics}
% Code listings
\usepackage{minted}
\providecommand{\mnt}[1]{\mintinline{python}{#1}}
\title{Geometry in analysis}
\author{}
\date{07 May 2022}
\begin{document}
\maketitle

\section*{Surfaces}

Surface can be given by a function ($U$ is usually open here):
\[ (x_1 \dots x_{n-1}) \in U \subset \R^{n-1}, f: U \to \R, 
\Gamma_f = \{ (x, y) \in \R^n \mid x \in U, y = f(x) \} \]

Coordinate lines are the images of sets $x_i = C, \forall i \in [n-1], C \in R$.

If we have path $\gamma$ in $U$ we can project it to the surface as $(\gamma, f \circ \gamma)$.

If $f$ is smooth at $x$ we can consider $y(x + h) = f(x) + \dd{f}\/(x)(h)$ and call it 
affine tangent space. And without $f(x)$ it's tangent space. Tangent vectors are those lying
in tangent space. Also these vectors are those being tangent (at $x$) to some curve in 
the surface.

Normal to the surface is $\bar n(x) = \begin{pmatrix}\nabla f(x)\\ -1\end{pmatrix}$.

And $f: \R^n \to \R^m$ are also possible.
\end{document}