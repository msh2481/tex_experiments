\documentclass[12pt]{article}

% Cyrillic support
\usepackage{polyglossia}
\setmainfont{XCharter}
\usepackage{xcharter-otf}
\usepackage{parskip}

% Images
\usepackage[dvips,xetex]{graphicx}
\graphicspath{ {./data/} }
% Spaces after sentence ends
\frenchspacing
% Advanced coloring
\usepackage{xcolor}
% Math environments
\usepackage{amsthm}
% \usepackage{amssymb}
\newtheorem{thr}{Th.}
\newtheorem{alg}{Alg.}
\theoremstyle{definition}
\newtheorem*{df}{Def}
% One-symbol defines with mathbb
\usepackage{amsfonts}
\providecommand{\N}{\mathbb N}
\providecommand{\Z}{\mathbb Z}
\providecommand{\P}{\mathbb P}
\providecommand{\R}{\mathbb R}
\providecommand{\C}{\mathbb C}
\providecommand{\F}{\mathbb F}
\providecommand{\Q}{\mathbb Q}
\providecommand{\bigO}{\mathcal O}
\providecommand{\go}[1]{\url{#1.pdf}}
\providecommand{\then}{\Rightarrow}
\providecommand{\from}{\Leftarrow}
\providecommand{\iff}{\Leftrightarrow}

\renewcommand{\N}{\mathbb N}
\renewcommand{\Z}{\mathbb Z}
\renewcommand{\P}{\mathbb P}
\renewcommand{\R}{\mathbb R}
\renewcommand{\C}{\mathbb C}
\renewcommand{\F}{\mathbb F}
\renewcommand{\Q}{\mathbb Q}
\renewcommand{\bigO}{\mathcal O}
\renewcommand{\go}[1]{\url{#1.pdf}}
\renewcommand{\then}{\Rightarrow}
\renewcommand{\from}{\Leftarrow}
\renewcommand{\iff}{\Leftrightarrow}

\renewcommand{\le}{\leqslant}
\renewcommand{\ge}{\geqslant}
\usepackage{physics}
% Code listings
\usepackage{minted}
\usepackage{hyperref}
\hypersetup{
	linkcolor=.,
	colorlinks=true,
	urlcolor=cyan
}
\renewcommand{\go}[1]{\href{./#1.pdf}{#1}}
\providecommand{\mnt}[1]{\mintinline{python}{#1}}
\title{Conditional extrema}
\author{}
\date{12 May 2022}
\begin{document}
\maketitle

\begin{df} $f: X \to \R, M \subset X, x_0 \in M$ --- conditional local minimum iff
$\exists \eps > 0: f(x_0) \leq f(x) \forall x \in B_{\eps}(x_0) \cap M$ \end{df}

Usually $M$ is defined by some condition, e.g. level set of a function, i.e. 
\[ \Phi: X \to \R, M = \{ x \in X \mid \Phi(x) = 0 \} \] 

Also $x_0$ should be an inner point, otherwise there are no differentials.

\subsection*{$\R^m$ case}

\[ f: \R^{m+n} \to \R, \Phi: \R^{m+n} \to \R \] 
\[ M = \{ z \in X \mid \Phi(z) = 0 \}, z \in M \iff \begin{cases}
    \Phi_1(z) = 0 \\
    \dots \\ 
    \Phi_m(z) = 0 \\
\end{cases} \]

\[ \dd{\Phi(z)} = \qty( \pdv{\Phi_i}{z_j} ) \]

\begin{df} M is nondegenerate in $z$ if $\rank \dd{\Phi(z)} = m$. \end{df}

In that case let's choose $m$ independent
columns and put them in the end. First $n$ coordinates now are called $x$ and last $m$ --- $y$.

\[ \dd{\Phi(z)} = \qty( \partial_x \Phi(x, y) \partial_y \Phi(x, y)) \]

By implicit function theorem [\go{OG6Z}] in a neighborhood $y = \varphi(x)$.
Now $z = (x_0, y_0)$ is a conditional minimum iff $x$ is a minimum of $\bar f(x) = f(x, \varphi(x)) \then \dd{\bar f(x_0)} = 0$.

\[ \partial_x f(x_0, \varphi(x_0)) + \partial_y f(x_0, \varphi(x_0)) \dd{\varphi(x_0)} = 0 \in \R^n \]
So it has a chance of giving a solution (because it removes $n$ degrees of freedom).

\[ \forall x \in B(x_0, \eps): \Phi(x, \varphi(x)) = 0 \then \partial_x \Phi(x_0, \varphi(x_0)) + \partial_y \Phi(x_0, \varphi(x_0)) \dd{\varphi(x_0)} = 0 \] 

\[ \dd{\varphi(x_0)} = (\partial_y \Phi(x_0, \varphi(x_0)))^{-1} \partial_x \Phi(x_0, \varphi(x_0)) \] 
\[  \partial_x f(x_0, y_0) + \partial_y f(x_0, y_0)  (\partial_y \Phi(x_0, \varphi(x_0)))^{-1} \partial_x \Phi(x_0, \varphi(x_0)) = 0  \]

\[ \lambda = \partial_y f(x_0, y_0)  (\partial_y \Phi(x_0, \varphi(x_0)))^{-1} \in (\R^m \to \R)\] 

\[ \begin{cases}
    \partial_x f(x_0, y_0) + \lambda \partial_x \Phi(x_0, \varphi(x_0)) = 0\\ 
    \partial_y f(x_0, y_0) + \lambda \partial_y \Phi(x_0, \varphi(x_0)) := 0\\
\end{cases} \]

Lagrange test:
\[ \begin{cases}
    \dd{f}\/(z_0) - \lambda \dd{\Phi}\/(z_0) = 0\\
    \Phi(z_0) = 0\\
\end{cases} \]

There are $m + n$ indeterminates in $z$ and $m$ in $\lambda$. But also there are $2m + n$ equations, so 
they might be enough to find $z, \lambda$.

\begin{thr} If $z_0$ is a local conditional extremum of $f$ when $\Phi = 0$ and $\Phi$ is nondegenerate in $z$ then 
    $\exists \lambda \in \R^m: \dd{(f - \lambda \Phi)(z_0)} = 0 $.
\end{thr}

\subsection*{Geometric meaning}
If $z_0$ is a local conditional minimum then value should increase along every direction in the surface.
$ \dd{f(z_0)} h = 0 $ when $h$ is tangent to the surface.

\[ \forall h \perp L = \dotp{\nabla \Phi_1, \dots \nabla \Phi_m}: \nabla f \perp h \] 
Because $(L^\perp)^\perp = L$ (in finite dimensional case, to prove do something with bases):
\[ \nabla f \in L \then \exists \nabla_1 \dots \nabla_m: \nabla f(z_0) = \lambda_1 \nabla \Phi_1(z_0) + \dots + \lambda_m \nabla \Phi_m(z_0) \]

\subsection*{Example: quadratic form on a sphere}

\[ \R^m, f(x) = \dotp{Ax, x}, A = A^T, Phi(x) = \norm{x}^2 - 1 = \dotp{x, x} - 1 \]
\[ \dd{f(x)} =  2\dotp{A \dd{x}, x}, \dd{\Phi(x)} = 2\dotp{x, \dd{x}} \] 
\[ \dd{f - \lambda \Phi} = 0 \then Ax = \lambda x \] 
Suppose we found $\lambda_1, v_1$:
\[ x \perp v_1 \then \lambda_1 \dotp{x, v_1} \then \dotp{x, Av_1} \then \dotp{Ax, v_1} \]
So consider $A_1 = A|_{\dotp{v_1}^\perp}, f_1 = f|_{\dotp{v_1}^\perp}$ and solve it with more restraints.

\subsection*{Didona's task}

\[ f(0) = f(1) = 0, J[f] = \int_0^1 f(x) dx \to \max, \Phi[f] = \int_0^1 \sqrt{1 + (f')^2} dx - l = 0 \]

Let's use Lagrange multipliers with partial derivatives:
\[ \exists \lambda \forall h, h(0) = h(1) = 0: \partial_h (J - \lambda \Phi)(f) = 0 \] 

\begin{proof}
Suppose $\exists h_0: \partial_{h_0} \Phi(f) \ne 0$. Choose $h$:
\[ g(t, s) = \Phi[f + th + sh_0] \] 
Why $\exists s: g(t, s) = 0$? If $\partial_s g(0, 0) \ne 0$ we have it from implicit function theorem.
\[ \partial_s \Phi[f + th + sh_0] = \dd{\Phi[f + th + sh_0]} h_0 = \partial_{h_0} (f + th + sh_0) \]
\[  \partial_{h_0} \Phi(f) \ne 0 \] 
so there is some $s(t)$ in a neighborhood.

\[ \varphi(t) = J(f + th = s(t)h_0) \then 0\text{ --- maximum point. }\] 
\[ \varphi'(0) = 0 \] 
\[ \partial_{h + s'(0) + h_0}J(f) = 0 \] 
\[ \partial_h J(f) + s'(0) \partial_{h_0} J(f) = 0 \] 
And from $\Phi[f + th + s(t)h_0] = 0$:
\[ \partial_h \Phi(f) + s'(0) \partial_{h_0} \Phi(f) = 0 \]
As in Lagrange method:
\[ \partial_h J(f) - \lambda \partial_h \Phi(f) = 0, \lambda \frac{\partial_{h_0} J(f)}{\partial_{h_0} \Phi(f)} \]
\end{proof}

\[ \partial_h (J - \lambda \Phi)(f) = 0 \] 
\[ \partial_h (\int_0^1 (f - \lambda\sqrt{1 + (f')^2}) dx) = 0 \]
\[ \partial_2 F(...) - \frac{\dd}{\dd{x}} \partial_3 F(...) = 0 \] 
\[ 1 - \frac{\dd}{\dd{x}}(-\lambda \frac{f'}{\sqrt{1 + (f')^2}}) = 0 \] 
\[ -\lambda \frac{f'}{\sqrt{1 + (f')^2}} = x + C \] 
\[ \frac{(f')^2}{1 + (f')^2} = \frac{x + C}{\lambda}^2 \]
\[ \frac{1}{(f')^2} = \frac{\lambda^2}{(x+C)^2} - 1 \]
\[ f'(x) = \sqrt{\frac{(x+C)^2}{\lambda^2 - (x+C)^2}} \] 
\[ f(x) = \int \frac{x + c}{\sqrt{\lambda^2 - (x+c)^2}} \] 
With $t = x + c$ we get:
\[ f(x) = \sqrt{\lambda^2 - (x+C)^2} + C_1 \] 
So it's a part of a circle with center in $(-\frac12, ...)$ and radius $\lambda$.


\end{document}