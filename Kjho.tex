\documentclass[12pt]{article}

% Cyrillic support
\usepackage{polyglossia}
\setmainfont{XCharter}
\usepackage{xcharter-otf}
\usepackage{parskip}

% Images
\usepackage[dvips,xetex]{graphicx}
\graphicspath{ {./data/} }
% Spaces after sentence ends
\frenchspacing
% Advanced coloring
\usepackage{xcolor}
\usepackage{hyperref}
\hypersetup{
	linkcolor=.,
	colorlinks=true,
	urlcolor=cyan
}
% Math environments
\usepackage{amsthm}
% \usepackage{amssymb}
\newtheorem{thr}{Th.}
\newtheorem{alg}{Alg.}
% One-symbol defines with mathbb
\usepackage{amsfonts}
\providecommand{\N}{\mathbb N}
\providecommand{\Z}{\mathbb Z}
\providecommand{\P}{\mathbb P}
\providecommand{\R}{\mathbb R}
\providecommand{\C}{\mathbb C}
\providecommand{\F}{\mathbb F}
\providecommand{\Q}{\mathbb Q}
\providecommand{\bigO}{\mathcal O}
\providecommand{\go}[1]{\url{#1.pdf}}
\renewcommand{\N}{\mathbb N}
\renewcommand{\Z}{\mathbb Z}
\renewcommand{\P}{\mathbb P}
\renewcommand{\R}{\mathbb R}
\renewcommand{\C}{\mathbb C}
\renewcommand{\F}{\mathbb F}
\renewcommand{\Q}{\mathbb Q}
\renewcommand{\bigO}{\mathcal O}
\renewcommand{\go}[1]{\url{#1.pdf}}
\usepackage{physics}
% Code listings
\usepackage{minted}
\providecommand{\mnt}[1]{\mintinline{python}{#1}}
\title{Euclidean space and Gram-Schmidt process}
\author{}
\date{10 May 2022}

\begin{document}
\maketitle

\subsection*{Euclidean space}

\begin{df} Euclidean space is $V, \langle . \rangle$, where
    $V$ is a vector space over $\R$ and $\langle . \rangle$ is a bilinear form
    called dot product. \end{df}

What we already know about it:
\[
    \norm{x} = \sqrt{\dotp{x, x}},\
    \rho(x, y) = \norm{x - y},\
    |\dotp{x, y}| \le \norm{x}\norm{y} \]
Also we can define cosine of angle between vectors:
\[ -1 \le \frac{\dotp{x, y}}{\norm{x}{y}} \le 1 \]
And projection:
\[ \proj_{v}: V \to \dotp{v}, \proj_{v}{x} = \frac{v}{\norm{v}} \norm{x} \cos \alpha =
    \frac{\dotp{x, v}}{\norm{v}^2} v \]
Or projection on $\dotp{v}^\perp$ by subtracting: $x - \proj_vx$.

\subsection*{Gram-Schmidt process}

\begin{df} Set of vectors $e_i$ in a Euclidean space
    is called orthnormal if $e_i \perp e_j$ and $\norm{e_i} = 1$ \end{df}

\begin{df} Orthogonalization is a process which for a set of vectors
    $e_i$ gives another set $f_i$, that are orthnormal and
    $\forall k: \dotp{e_1 \dots e_k} = \dotp{f_1 \dots f_k}$ \end{df}

Necessarily $f_1 = \frac{e_1}{\norm{e_1}}$, because 
$\dotp{f_1} = \dotp{e_1}, \norm{f_1} = 1$.

Now, $f_i \perp \dotp{e_1\dots e_{i-1}} \iff f_i \perp \dotp{f_1 \dots f_{i-1}}$.
Also it should contain something about $e_i$ in it, so we can just subtract all projections: 
\[f_i = e_i - \sum \proj_{f_j}{e_i}\]
Then just normalize it.

\end{document}