\documentclass[12pt]{article}

% Cyrillic support
\usepackage{polyglossia}
\setmainfont{XCharter}
\usepackage{xcharter-otf}
\usepackage{parskip}

% Images
\usepackage[dvips,xetex]{graphicx}
\graphicspath{ {./data/} }
% Spaces after sentence ends
\frenchspacing
% Advanced coloring
\usepackage{xcolor}
\usepackage{hyperref}
\hypersetup{
	linkcolor=.,
	colorlinks=true,
	urlcolor=cyan
}
% Math environments
\usepackage{amsthm}
% \usepackage{amssymb}
\newtheorem{thr}{Th.}
\newtheorem{alg}{Alg.}
% One-symbol defines with mathbb
\usepackage{amsfonts}
\providecommand{\N}{\mathbb N}
\providecommand{\Z}{\mathbb Z}
\providecommand{\P}{\mathbb P}
\providecommand{\R}{\mathbb R}
\providecommand{\C}{\mathbb C}
\providecommand{\F}{\mathbb F}
\providecommand{\Q}{\mathbb Q}
\providecommand{\bigO}{\mathcal O}
\providecommand{\go}[1]{\url{#1.pdf}}
\renewcommand{\N}{\mathbb N}
\renewcommand{\Z}{\mathbb Z}
\renewcommand{\P}{\mathbb P}
\renewcommand{\R}{\mathbb R}
\renewcommand{\C}{\mathbb C}
\renewcommand{\F}{\mathbb F}
\renewcommand{\Q}{\mathbb Q}
\renewcommand{\bigO}{\mathcal O}
\renewcommand{\go}[1]{\url{#1.pdf}}
\usepackage{physics}
% Code listings
\usepackage{minted}
\providecommand{\mnt}[1]{\mintinline{python}{#1}}
\title{}
\author{}
\date{16 May 2022}
\begin{document}
\maketitle

$V$ --- Euclidean space. $L: V \to V$. We can transform it to a bilinear form in two ways:
\[ \dotp{Lu, v} \text{ or } \dotp{u, Lv} \] 

When these two are equal, $L$ is called self-adjoint (i.e. $L = L*$).

\begin{df} Adjoint operator $L*$ for $L$ is the unique operator such that $\dotp{u, Lv} = \dotp{L*u, v}$ \end{df}

\section*{Hermitian matrices}

\begin{thr}
    $A$ --- self-adjoint operator on $V$ then eigenvalues of $A$ are real.
\end{thr}
\begin{proof}
Consider $(\lambda, v)$.
\[ \dotp{v, Av} = \dotp{v, \lambda v} = \lambda \norm{v}^2 \] 
\[ \dotp{v, Av} = \dotp{Av, v} = \dotp{\lambda v, v} = \bar \lambda \norm{v}^2 \]
\[ \lambda = \bar \lambda \then \lambda \in \R \] 
\end{proof} 

\begin{thr}
    $A$ --- self-adjoint operator, then exists orthonormal basis of its eigenvectors.
\end{thr}
\begin{proof}
    Consider $(\lambda, v)$, and restrict $A$ on $\dotp{v}^\perp$. It's an invariant space, because 
    $u \in \dotp{v}^\perp \then \dotp{Au, v} = \dotp{u, Av} = \dotp{u, \lambda v} = 0 $.
    Now $A|_{\dotp{v}^\perp}$ is still self-adjoint, so we can continue to build the basis iteratively.
\end{proof}

\begin{thr}
    If $Q$ is orthogonal $n \times n$ matrix on $\R$, its eigenvalues have absolute value 1.
\end{thr}
\begin{proof}
    Let's work in $\bar Q^T Q = E_n$.
    Again proof that $\dotp{v}^\perp$ is invariant. Then we have an orthonormal basis,
    so in it $Q$ is a diagonal matrix, and from $\bar Q^T Q = E_n$ we get $\bar \lambda_i \lambda_i = 1$.
\end{proof}

\begin{df} $L$ is called normal when $LL* = L*L$. \end{df}

\end{document}